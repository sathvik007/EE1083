\documentclass[journal,12pt,twocolumn]{IEEEtran}
%
\usepackage{setspace}
\usepackage{gensymb}
\usepackage{xcolor}
\usepackage{caption}
%\usepackage{subcaption}
%\doublespacing
\singlespacing

%\usepackage{graphicx}
%\usepackage{amssymb}
%\usepackage{relsize}
\usepackage[cmex10]{amsmath}
\usepackage{mathtools}
%\usepackage{amsthm}
%\interdisplaylinepenalty=2500
%\savesymbol{iint}
%\usepackage{txfonts}
%\restoresymbol{TXF}{iint}
%\usepackage{wasysym}
\usepackage{amsthm}
\usepackage{mathrsfs}
\usepackage{txfonts}
\usepackage{stfloats}
\usepackage{cite}
\usepackage{cases}
\usepackage{subfig}
%\usepackage{hyperref}
%\usepackage{xtab}
\usepackage{longtable}
\usepackage{multirow}
%\usepackage{algorithm}
%\usepackage{algpseudocode}
\usepackage{enumitem}
\usepackage{mathtools}
\usepackage{iithtlc}
%\usepackage[framemethod=tikz]{mdframed}
\usepackage{listings}
\usepackage{listings}
    \usepackage[latin1]{inputenc}                                 %%
    \usepackage{color}                                            %%
    \usepackage{array}                                            %%
    \usepackage{longtable}                                        %%
    \usepackage{calc}                                             %%
    \usepackage{multirow}                                         %%
    \usepackage{hhline}                                           %%
    \usepackage{ifthen}                                           %%
  %optionally (for landscape tables embedded in another document): %%
    \usepackage{lscape}     



%\usepackage{stmaryrd}


%\usepackage{wasysym}
%\newcounter{MYtempeqncnt}
\DeclareMathOperator*{\Res}{Res}
%\renewcommand{\baselinestretch}{2}
\renewcommand\thesection{\arabic{section}}
\renewcommand\thesubsection{\thesection.\arabic{subsection}}
\renewcommand\thesubsubsection{\thesubsection.\arabic{subsubsection}}

\renewcommand\thesectiondis{\arabic{section}}
\renewcommand\thesubsectiondis{\thesectiondis.\arabic{subsection}}
\renewcommand\thesubsubsectiondis{\thesubsectiondis.\arabic{subsubsection}}

% correct bad hyphenation here
\hyphenation{op-tical net-works semi-conduc-tor}

%\lstset{
%language=C,
%frame=single, 
%breaklines=true
%}

%\lstset{
	%%basicstyle=\small\ttfamily\bfseries,
	%%numberstyle=\small\ttfamily,
	%language=Octave,
	%backgroundcolor=\color{white},
	%%frame=single,
	%%keywordstyle=\bfseries,
	%%breaklines=true,
	%%showstringspaces=false,
	%%xleftmargin=-10mm,
	%%aboveskip=-1mm,
	%%belowskip=0mm
%}

%\surroundwithmdframed[width=\columnwidth]{lstlisting}
\def\inputGnumericTable{}                                 %%
\definecolor{mauve}{rgb}{0.58,0,0.82}
\definecolor{dkgreen}{rgb}{0,0.6,0}
\definecolor{cyan}{rgb}{0.0,0.6,0.6}
\lstset{
language=Python,
frame=single, 
breaklines=true
commentstyle=\color{dkgreen},
stringstyle=\color{mauve},
keywordstyle=\color{cyan},}
 

\begin{document}
%

\theoremstyle{definition}
\newtheorem{theorem}{Theorem}[section]
\newtheorem{problem}{Problem}
\newtheorem{proposition}{Proposition}[section]
\newtheorem{lemma}{Lemma}[section]
\newtheorem{corollary}[theorem]{Corollary}
\newtheorem{example}{Example}[section]
\newtheorem{definition}{Definition}[section]
%\newtheorem{algorithm}{Algorithm}[section]
%\newtheorem{cor}{Corollary}
\newcommand{\BEQA}{\begin{eqnarray}}
\newcommand{\EEQA}{\end{eqnarray}}
\newcommand{\define}{\stackrel{\triangle}{=}}

\bibliographystyle{IEEEtran}
%\bibliographystyle{ieeetr}

\providecommand{\nCr}[2]{\,^{#1}C_{#2}} % nCr
\providecommand{\nPr}[2]{\,^{#1}P_{#2}} % nPr
\providecommand{\mbf}{\mathbf}
\providecommand{\pr}[1]{\ensuremath{\Pr\left(#1\right)}}
\providecommand{\qfunc}[1]{\ensuremath{Q\left(#1\right)}}
\providecommand{\sbrak}[1]{\ensuremath{{}\left[#1\right]}}
\providecommand{\lsbrak}[1]{\ensuremath{{}\left[#1\right.}}
\providecommand{\rsbrak}[1]{\ensuremath{{}\left.#1\right]}}
\providecommand{\brak}[1]{\ensuremath{\left(#1\right)}}
\providecommand{\lbrak}[1]{\ensuremath{\left(#1\right.}}
\providecommand{\rbrak}[1]{\ensuremath{\left.#1\right)}}
\providecommand{\cbrak}[1]{\ensuremath{\left\{#1\right\}}}
\providecommand{\lcbrak}[1]{\ensuremath{\left\{#1\right.}}
\providecommand{\rcbrak}[1]{\ensuremath{\left.#1\right\}}}
\theoremstyle{remark}
\newtheorem{rem}{Remark}
\newcommand{\sgn}{\mathop{\mathrm{sgn}}}
\providecommand{\abs}[1]{\left\vert#1\right\vert}
\providecommand{\res}[1]{\Res\displaylimits_{#1}} 
\providecommand{\norm}[1]{\lVert#1\rVert}
\providecommand{\mtx}[1]{\mathbf{#1}}
\providecommand{\mean}[1]{E\left[ #1 \right]}
\providecommand{\fourier}{\overset{\mathcal{F}}{ \rightleftharpoons}}
%\providecommand{\hilbert}{\overset{\mathcal{H}}{ \rightleftharpoons}}
\providecommand{\system}{\overset{\mathcal{H}}{ \longleftrightarrow}}
	%\newcommand{\solution}[2]{\textbf{Solution:}{#1}}
\newcommand{\solution}{\noindent \textbf{Solution: }}
\providecommand{\dec}[2]{\ensuremath{\overset{#1}{\underset{#2}{\gtrless}}}}
%\numberwithin{equation}{subsection}
\numberwithin{equation}{problem}
%\numberwithin{problem}{subsection}
%\numberwithin{definition}{subsection}
\makeatletter
\@addtoreset{figure}{problem}
\makeatother

\let\StandardTheFigure\thefigure
%\renewcommand{\thefigure}{\theproblem.\arabic{figure}}
\renewcommand{\thefigure}{\theproblem}


%\numberwithin{figure}{subsection}

%\numberwithin{equation}{subsection}
%\numberwithin{equation}{section}
%%\numberwithin{equation}{problem}
%%\numberwithin{problem}{subsection}
\numberwithin{problem}{section}
%%\numberwithin{definition}{subsection}
%\makeatletter
%\@addtoreset{figure}{problem}
%\makeatother
\makeatletter
\@addtoreset{table}{problem}
\makeatother

\let\StandardTheFigure\thefigure
\let\StandardTheTable\thetable
%%\renewcommand{\thefigure}{\theproblem.\arabic{figure}}
%\renewcommand{\thefigure}{\theproblem}
\renewcommand{\thetable}{\theproblem}
%%\numberwithin{figure}{section}

%%\numberwithin{figure}{subsection}



\def\putbox#1#2#3{\makebox[0in][l]{\makebox[#1][l]{}\raisebox{\baselineskip}[0in][0in]{\raisebox{#2}[0in][0in]{#3}}}}
     \def\rightbox#1{\makebox[0in][r]{#1}}
     \def\centbox#1{\makebox[0in]{#1}}
     \def\topbox#1{\raisebox{-\baselineskip}[0in][0in]{#1}}
     \def\midbox#1{\raisebox{-0.5\baselineskip}[0in][0in]{#1}}

\vspace{3cm}

\title{ 
	\logo{
Databases through Python-Flask and MariaDB
	}
}



% paper title
% can use linebreaks \\ within to get better formatting as desired
%\title{Installation of Python-Flask and Mariadb.}
%
%
% author names and IEEE memberships
% note positions of commas and nonbreaking spaces ( ~ ) LaTeX will not break
% a structure at a ~ so this keeps an author's name from being broken across
% two lines.
% use \thanks{} to gain access to the first footnote area
% a separate \thanks must be used for each paragraph as LaTeX2e's \thanks
% was not built to handle multiple paragraphs
%

\author{Tanmay Agarwal, Durga Keerthi and G V V Sharma$^{*}$% <-this % stops a space
\thanks{Tanmay is an intern with the TLC, IIT Hyderabad.  Durga is a UG student at IIT Hyderabad.  *GVV Sharma is with the Department
of Electrical Engineering, Indian Institute of Technology, Hyderabad
502285 India e-mail:  gadepall@iith.ac.in. All content in this manual is released under GNU GPL.  Free and open source.}% <-this % stops a space
%\thanks{J. Doe and J. Doe are with Anonymous University.}% <-this % stops a space
%\thanks{Manuscript received April 19, 2005; revised January 11, 2007.}}
}
% note the % following the last \IEEEmembership and also \thanks - 
% these prevent an unwanted space from occurring between the last author name
% and the end of the author line. i.e., if you had this:
% 
% \author{....lastname \thanks{...} \thanks{...} }
%                     ^------------^------------^----Do not want these spaces!
%
% a space would be appended to the last name and could cause every name on that
% line to be shifted left slightly. This is one of those "LaTeX things". For
% instance, "\textbf{A} \textbf{B}" will typeset as "A B" not "AB". To get
% "AB" then you have to do: "\textbf{A}\textbf{B}"
% \thanks is no different in this regard, so shield the last } of each \thanks
% that ends a line with a % and do not let a space in before the next \thanks.
% Spaces after \IEEEmembership other than the last one are OK (and needed) as
% you are supposed to have spaces between the names. For what it is worth,
% this is a minor point as most people would not even notice if the said evil
% space somehow managed to creep in.



% The paper headers
%\markboth{Journal of \LaTeX\ Class Files,~Vol.~6, No.~1, January~2007}%
%{Shell \MakeLowercase{\textit{et al.}}: Bare Demo of IEEEtran.cls for Journals}
% The only time the second header will appear is for the odd numbered pages
% after the title page when using the twoside option.
% 
% *** Note that you probably will NOT want to include the author's ***
% *** name in the headers of peer review papers.                   ***
% You can use \ifCLASSOPTIONpeerreview for conditional compilation here if
% you desire.




% If you want to put a publisher's ID mark on the page you can do it like
% this:
%\IEEEpubid{0000--0000/00\$00.00~\copyright~2007 IEEE}
% Remember, if you use this you must call \IEEEpubidadjcol in the second
% column for its text to clear the IEEEpubid mark.



% make the title area
\maketitle

\tableofcontents

\bigskip

\begin{abstract}
%\boldmath
Databases software applications for small establishments like schools, shops, etc.. can be easily built using the MariaDB database, Python-Flask connector and HTML.  This manual shows how to install these  free software tools and build a simple application using them.
\end{abstract}
% IEEEtran.cls defaults to using nonbold math in the Abstract.
% This preserves the distinction between vectors and scalars. However,
% if the journal you are submitting to favors bold math in the abstract,
% then you can use LaTeX's standard command \boldmath at the very start
% of the abstract to achieve this. Many IEEE journals frown on math
% in the abstract anyway.

% Note that keywords are not normally used for peerreview papers.
%\begin{IEEEkeywords}
%Cooperative diversity, decode and forward, piecewise linear
%\end{IEEEkeywords}



% For peer review papers, you can put extra information on the cover
% page as needed:
% \ifCLASSOPTIONpeerreview
% \begin{center} \bfseries EDICS Category: 3-BBND \end{center}
% \fi
%
% For peerreview papers, this IEEEtran command inserts a page break and
% creates the second title. It will be ignored for other modes.
\IEEEpeerreviewmaketitle


%\newpage
%\section{Component Pin Diagrams}
%%
%\input{chapter1}
%

%\newpage
\section{Python-flask}

 Flask is Python framework for creating web applications.
\subsection{Installation}
\begin{enumerate}
\item Run the following commands on the terminal
\begin{lstlisting}
sudo apt-get update
sudo apt-get install python-pip
sudo pip install flask
sudo pip install mysql-connector
\end{lstlisting}
\end{enumerate}

\subsection{Testing Flask}
Since installation of Flask is now complete, verify that flask is working by using the example below. 
\begin{enumerate}
 
\item \textbf{Code:}

\begin{description}
\item \lstinputlisting[language=Python]{./codes/hello.py}
\end{description}

\item Save the file as \textbf{hello.py}.

\item open the terminal and run 
\begin{lstlisting}
python hello.py
\end{lstlisting}
An ip address will be displayed on the terminal.

\item Open the address on your favourite browser. "Hello world" will be displayed
\end{enumerate}






%\input{./figs/components.tex}
\section{Mariadb }
MariaDB Server is one of the most popular database servers in the world. The following installation instructions are for Ubuntu.  Installation on other Linux systems are likely to be similar.
\subsection{\textbf{Software Installation}}
Refer to Link
\\
https://www.liquidweb.com/kb/how-to-install-mariadb-5-5-on-ubuntu-14-04-lts/
\begin{enumerate}

\item Type the following commands on the terminal
%Add MariaDB Respository.
\begin{lstlisting}
sudo apt-get install software-properties-common
sudo apt-key adv --recv-keys --keyserver hkp://keyserver.ubuntu.com:80 0xcbcb082a1bb943db
sudo add-apt-repository 'deb http://mirror.jmu.edu/pub/mariadb/repo/5.5/ubuntu trusty main'
sudo apt-get update
sudo apt-get install mariadb-server
\end{lstlisting}

%^The software-properties-common package should already be installed, but just in case:
%
%Command:- \textbf{sudo apt-get install software-properties-common}

%\item Import the MariaDB public key used by the package management system:
%
%command:\textbf{sudo apt-key adv --recv-keys --keyserver hkp://keyserver.ubuntu.com:80 0xcbcb082a1bb943db}

%\item Add the MariaDB Repository:

%command: \textbf{sudo add-apt-repository 'deb http://mirror.jmu.edu/pub/mariadb/repo/5.5/ubuntu trusty main'}
%
%Now reload the package database.

%command:\textbf{sudo apt-get update}
%\item Install MariaDB.
%
%command: \textbf{sudo apt-get install mariadb-server}
%
You may receive the following prompt or something similar:

After this operation, 116 MB of additional disk space will be used.
Do you want to continue? [Y/n]

Enter Y to continue.

Next you will be asked:

New password for the MariaDB root user:

This is an administrative account in MariaDB with elevated privileges; enter a strong password.

Then you will be asked to verify the root MariaDB password:

Repeat password for the MariaDB root user:

That is it! Your basic MariaDB installation is now complete!

Be sure to stop MariaDB before proceeding to the next step:

\textbf{sudo service mysql stop}

\end{enumerate}
\subsection{Configuration}
Configure and Secure MariaDB for Use
\begin{enumerate}
\item 

Now we will instruct MariaDB to create its database directory structure:
\\
\textbf{sudo mysql\_install\_db}

\item Start MariaDB
\\
\textbf{sudo service mysql start}

\item And now let us secure MariaDB by removing the test databases and anonymous user created by default:

\textbf{sudo mysql\_secure\_installation}

\item You will be prompted to enter your current password. Enter the root MariaDB password set during installation:
\\
Enter current password for root (enter for none):
\\
Then, assuming you set a strong root password, go ahead and enter n at the following prompt:
\\
Change the root password? [Y/n] n
\\
Remove anonymous users, Y:
\\
Remove anonymous users? [Y/n] Y
\\
Disallow root logins remotely, Y:
\\
Disallow root login remotely? [Y/n] Y
\\
Remove test database and access to it, Y:
\\
Remove test database and access to it? [Y/n] Y
\\
And reload privilege tables, Y:
\\
Reload privilege tables now? [Y/n] Y

\item Verify MariaDB Installation

 Check Version
\\
\textbf{mysql -V}
\end{enumerate}

\section{Database Application}
\subsection{Creating a Database}
%\subsection{Getting Started With MariaDB}
\begin{enumerate}
\item 
%Creating a Databace in Mysql and Mariadb.
Open the terminal and type
\begin{lstlisting}
mysql -u root -p
\end{lstlisting}
You will be asked for a password. Enter it.   
\item Create a database called test using the following command.
\begin{lstlisting}
CREATE DATABASE Test;
\end{lstlisting}
\item In order to use the Database type
\begin{lstlisting}
USE Test;
\end{lstlisting}
You will enter into the Database called Test.
\item Now create a table named test with parameters as Name and Roll Number.
\begin{lstlisting}
CREATE TABLE test(name varchar(20) not null,roll varchar(20) not null);
\end{lstlisting}
varchar(20) means string of size 20 characters.
%\item Creating a table in database
%\\
%command:\textbf{CREATE TABLE table\_name(name varchar(20) not null, roll varchar(20) not null);}
%\\
%Table will be created in Database.
%
%varchar = It is data type. You can also use INT,CHAR,DATE etc.
%\\
%Note:- The number given along with varchar is the the maximum size a string can have.
%\\
%To see the format of the fields in table type:
%\\
%\textbf{desc table\_name;}
%\\
\item To see the format of the fields in \textbf{test}
\begin{lstlisting}
desc test;
\end{lstlisting}
%\\
%command:\textbf{Show tables;}
\end{enumerate}


%Now we will make a simple example in which we will be giving three operation Storing, Displaying and Updating of the data.First we will see each operation one by one then all together.

\subsection{Creating HTML Forms}
%\subsection{Storing the Data into the Database}
%\begin{enumerate}
%\item In order to store the data in Database we first have to create a database.Say let the name of database is Test.Just enter into your mysql by command mention above.
%
%\item Creating a Database.
%command: \textbf{CREATE DATABASE Test;}

%\item Use that Database and 
%
%\item Your Database will be created.
%\item Now we will create the following files.\textbf{student.html,message.html and store.py}.Just open any text editor and type the codes given below.
%\\
%Note:- Make a template folder in same place where you are keeping your Python file and keep all the HTML file in it.

%\begin{description}
\begin{enumerate}
\item Type the following code in a file called \textbf{student.html} and open it using a browser.  You will see boxes with \textbf{Name, Roll}.  Also, there will be a button called \textbf{submit} and two links titled \textbf{Show List} and \textbf{Update}.  the 
\lstinputlisting[language=HTML]{./codes/student.html}

%Save this file as student.html.This File will display the Form to Enter the name and Roll Number of the student.
%\end{description}

%\begin{description}
\item Type the following code in a file called \textbf{message.html}.  The purpose of this file is to display status messages.
\lstinputlisting[language=HTML]{./codes/message.html}
%Save this file with message.html. 
%\end{description}
\item Save both the html files in a folder called \textbf{templates}.
\end{enumerate}
\subsection{Python Connector from Browser to Database}
\begin{enumerate}
\item Type the following code in a file called \textbf{store.py}.
\lstinputlisting[language=Python]{./codes/store.py}
%Save this file with store.py.This file act as a connection between the web page Student.html and the Database.This will help in storing the Data in database.
%\end{description}
\item Make sure that the python file is outside the \textbf{templates} directory.  Now type
\begin{lstlisting}
python store.py
\end{lstlisting}
on the terminal. An address will be displayed on the terminal.
\item Enter the above address in a browser.  Fill the name and roll number and hit submit.
%\item As soon as you hit the submit buttom message will be displayed Data has been stored.
\end{enumerate}

\subsection{Fetching the stored Data from the Database}
\begin{enumerate}
%\item Now we want to fetch the data that has been stored into the database.In order to this we will make one more HTML file(display.html).This file will Help to display the fetched data in a way that we want.
%\item 
%\begin{description}
\item Save the following code in a file called \textbf{display.html}.
\lstinputlisting[language=HTML]{./codes/display.html}
%\end{description}

\item %Make a python file that connect database and fetch all the Value from the particular table.
Save the following code in a file titled  \textbf{display.py}.    
%\begin{description}
\item \lstinputlisting[language=Python]{./codes/display.py}
%\end{description}

\item Now open the terminal and type 
\begin{lstlisting}
python display.py
\end{lstlisting}
An address will be displayed.
\item Open this address in a browser. You can see all the Name and Roll No entries in the database.
\end{enumerate}
\subsection{Updating the Database}
\begin{enumerate}
\item 
%Now we will create one more python file and Html file in order to update the already stored data.
    
%\begin{description}
\item Save the following code in a file with titled \textbf{show.html}.
\lstinputlisting[language=HTML]{./codes/show.html}

%\end{description}
\item Save the following code in a file titled \textbf{update.py}.
%begin{description}
\item \lstinputlisting[language=Python]{./codes/update.py}
%\end{description}
\item Now open the terminal and run the \textbf{update.py} file.
\item Update whatever data you wish to and click the Update button.
\item Run \textbf{display.py} to verify that your data is indeed updated.
%Here now you can update whatever you want to update. Just change the thing and click on update button. Data will be updated.
\end{enumerate}
\subsection{Linking all modules to create the Database application}
\begin{enumerate}
%\item Now we combine all the code in one file and will link them with each other.we will make onehtml file output.html and main python file that is app.py.
%\item For html file write below code:-
%\begin{description}
\item Save the following code in a file called \textbf{output.html}.

\lstinputlisting[language=HTML]{./codes/output.html}
%\end{description}
\item Save the following code in a file titled \textbf{app.py}
%\begin{description}
\lstinputlisting[language=Python]{./codes/app.py}
%\end{description}
\item Run \textbf{app.py}
\item Start using your application.
\item Modify your application so that you may delete a record.
\end{enumerate}


%\input{./chapters/chapter1}
%
%\section{Display Control through Arduino Software}
%\input{./chapters/chapter2}
%%
%\section{Decade Counter through Arduino}
%\input{./chapters/chapter3}
%%%
%\section{Karnaugh Maps}
%\input{./chapters/chapter4}
%%
%\section{Sequential Logic}
%\input{./chapters/chapter5}
%
%\section{C Programming}
%\input{./chapters/chapter6}

%\input{arduinoport}

%\bibliography{IEEEabrv,gvv_matrix}

%\input{chapter2} 
%%
%\newpage
%\section{$M$-ary Modulation}
%\input{chapter3} 
%
%\newpage
%\section{BER in Rayleigh Flat Slowly Fading Channels}
%\input{chapter4} 

\end{document}

